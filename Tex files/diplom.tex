\documentclass[12pt]{article}

\bibliographystyle{plain}

% Russian-specific packeges
%--------------------------------
\usepackage[T2A]{fontenc}
\usepackage[utf8]{inputenc}
\usepackage[russian]{babel}
%--------------------------------
\usepackage{csquotes}
\usepackage{hyphenat}

\usepackage{amsmath}
\usepackage{indentfirst}
\usepackage{graphicx}

\graphicspath{ {diploma_pictures} }

\usepackage{physics}


\begin{document}
    \tableofcontents

    \section{Введение}

    %===========================================================================================
    \subsection{Квантовые вычисления}
    Вычисления издавна сопутствуют человечеству. Ещё в период верхнего
    палеолита люди считали - на пальцах. В дальнейшем с ростом потребностей
    человечества наши предки столкнулись с необходимостью создания
    более сложной вычислительной техники. Появление земельной собственности
    требовало определения способов вычисления площади, развитие торговли
    требовало учёта товаров и денежных сумм, появилась потребность в
    средствах навигации и измерения времени. Вначале человек изобретал
    примитивные инструменты: уже упомянутый пальцевый счёт, счёт на камнях,
    насечки, узелковое письмо и т.д. Но примерно с \mbox{XV в.} в руках человека
    стали появляться существенно более сложные приспособления: палочки Непера,
    логарифмические линейки, \enquote{Паскалина}. И вот, наконец, в середине
    прошлого столетия был изобретён компьютер.

    Появление компьютера ознаменовало переход человечества на совершенно
    иной уровень. Например, одна из первых машин - аналоговый компьютер,
    разработанный ещё в 1927 году Вэниваром Бушем - уже позволяла решать
    дифференциальные уравнения второго порядка! Впоследствии техника очень
    быстро развивалась: были изобретены интегральные схемы, появился первый
    компьютер на транзисторах. Столь стремительный темп развития был отмечен
    в 1965 году Гордоном Эрлом Муром, соучередителем компании Intel, - он
    заметил, что число транзисторов, размещённых на кристалле
    интегральной схемы, удваевается каждые два года. До сегодняшнего дня
    этот закон, называемый \textit{\textquote*{заоном Мура}}, неплохо
    выполнялся.

    Но у всего есть свои пределы. Рост эффективности вычислительной техники
    достигался в основном за счёт миниатюризации элементов электроники.
    К сожалению, это уменьшение не может продолжаться вечно - на слишком малых
    масштабах начнают проявляться существенно квантовые эффекты, мешающие
    нормальной работе классических электронных элементов. На сегодняшний
    момент человечество уже вплотную подошло к таким масштабам.

    Естественным решением сложившейся ситуации видится переход к вычислительной
    парадигме, основанной на законах квантовой механики. Теория показывает, что
    такой квантовый подход в ряде задач намного превосходит по эффективности
    классический аналог. Например, в \mbox{1994 г.} Питер Шор показал, что задача поиска
    простых сомножителей целого числа может быть эффективно решена с помощью
    квантовых вычислений (алгоритм Шора имеет сложность $\order{\log[3](N)}$,
    в то время как лидирующий среди классических алгоритмов - специальный метод
    решета числового поля - имеет сложность
    $\exp{(1+o(1)) (\frac{32}{9}\log{N})^{1/3} (\log{\log{N}})^{2/3})}$,
    где N - факторизуемое число. По мнению многих исследователей никакой
    мыслимый прогресс классических вычислений не сможет преодолеть разрыва
    в производительности между классическим и квантовым компьютерами.

    Главным элементом квантового компьютера является \textit{кубит}
    (от анлг. \textit{qubit - \textbf{qu}antum \textbf{bit}}). Это двухуровневая
    квантовомеханическая система, аналог классического бита, способная занимать
    не только состояния \textquote{0} и \textquote{1}, но и все промежуточные состояния.
    Такие состояния называют состояниями суперпозиции и записывают как
    $\ket{\Psi} = \alpha \ket{0} +\beta \ket{1}$, где $\ket{0}$ и $\ket{1}$ - базисные
    состояния, соответствующие классическим \textquote{0} и \textquote{1}, а
    $\alpha,\beta$ - комплексные коэффициенты. Удобно изображать кубиты в виде
    сферы Блоха.

   

    Главная задача квантовых инженеров заключается в разработке
    надёжных средств управления кубитами: средств приготовления,
    преобразования и считывания их состояний. 
    %===========================================================================================


    %===========================================================================================
    \subsection{Холодные атомы}
    На данный момент в мире существует несколько перспективных подходов
    к реализации кубитов: холодные атомы, холодные ионы и сверхпроводники.
    Наша лаборатория работает с холодными атомами, а именно, с рубидием-87.

    Основная идея построения квантового компютера на рубидии, как, впрочем, и
    на других атомах, заключается в использовании двух сверхтонких подуровней
    его основного уровня ($5^2S_{\frac12}$). Эти состояния достаточно удалены
    от остальных, и поэтому атом можно рассматривать как двухуровневую
    квантовую систему, т.е. как кубит.
    
    Каким же образом управлять состоянием такого кубита? Квантовая механика
    показывает, что двухуровневая система периодически меняет своё состояние
    под воздействием электромагнитного излучения (\textit{осцилляции Раби}).
    Это легко визуализировать при помощи сферы Блоха: каждому состоянию
    кубита соответсвует некоторый вектор, направленный из центра сферы, при
    воздействии излучения этот вектор начинает прецессировать вокруг оси,
    напрвление которой задаётся параметрами излучения (частотой и интенсивностью).
    Например, если кубит, находящийся в состоянии $\ket{0}$ (северный полюс сферы Блоха),
    облучается пучком, вращающим вектор состояния вокруг оси $Ox$, то выждав определённое
    время, мы получим атом в состоянии $\ket{1}$ (южный полюс сферы Блоха). Импульс излучения,
    вращающий вектор состояния на 180 градусов, называется
    \textit{$\pi$-импульсом}.
    
    Для того, чтобы атом был действительно двухуровневой системой, его нужно
    изолировать и от других атомов. Для этого используются камеры со сверхвысоким
    вакуумом, в которых будущие кубиты удерживаются в подвешенном состоянии при
    помощи дипольных оптических ловушек - сфокусированных лазерных пучков,
    способных удерживать нейтральные атомы.

    Потенциал, создаваемый световым полем может быть выражен формулой
    \cite{grimm2000optical}:
    \begin{equation}
        U_{dip}(\textbf{r}) = -\frac{3\pi c^2}{2\omega_0^3}
            \left(\frac{\Gamma}{\omega_0 - \omega} +
                  \frac{\Gamma}{\omega_0 + \omega}\right)I(\textbf{r}),
        \label{dip_potential}
    \end{equation}
    где $\omega_0$ - частота рабочего перехода в атоме, $\omega$ - частота
    излучения лазера, $\Gamma$ - темп распада, $I(\textbf{r})$ - интенсивность.
    Введём обозначение $\Delta \equiv \omega - \omega_0$. Тогда , приняв во внимание,
    что в экспериментах обычно выполнено сильное неравенсто
    $\omega_0 - \omega \ll \omega_0 + \omega$, получим следующее приближение
    \begin{equation}
        U_{dip}(\textbf{r}) = \frac{3\pi c^2}{2\omega_0^3}
            \frac{\Gamma}{\Delta}I(\textbf{r}),
        \label{dip_pot}
    \end{equation}


    Любопытно, выражение (\ref{dip_pot}) показывает, что для $\Delta > 0$
    (\textit{\enquote{синяя} отстройка}) потенциал будет отталкивающим, а для
    $\Delta < 0$ (\textit{\enquote{красная} отстройка}) - притягивающим.
    Обычно используется второй вариант. Он легче реализуется (достаточно
    сфокусировать лазерный пучок в узкую перетяжку, см. рис. 
    и требует меньшей мощности. Однако, \enquote{красные} ловушки обладают рядом
    недостатков. Главный из них - интенсивное рассеяние фотонов: пленённый атом
    находится в потоке квантов света и, естественно, рассеивает их. Такое
    взаимодействие, во-первых, портит приготовленное состояние кубита,
    а во-вторых, рано или поздно, приводит к выбиванию атома из ловушки.
    Кроме этого, пребывание в интенсивном  потоке излучения приводит к
    \textit{штарковскому сдвигу} уровней, что усложняет работу с кубитом.

    \enquote{Синие} ловушки могут помочь избежать этих проблем. Для того чтобы
    захватить атом в \enquote{синюю} ловушку, требуется создать область окружённую
    со всех сторон светом с положительной отстройкой ($\Delta > 0$). Атом, таким
    образом, будет удерживаться в области тени, отталкиваясь от стенок ловушки.
    При таком подходе частица находится в области с низкой интенсивностью, а значит,
    рассеивает меньше фотонов.
    

    Данная работа направлена на исследование способов генерации \enquote{синих}
    ловушек и изучение характеристик кубитов, созданных на их основе.
    %===========================================================================================


    %===========================================================================================
    \section{Литературный обзор}
        В этом параграфе будет рассмотрен только один тип \enquote{синих} ловушек:
        \enquote{бутылочные}. С ловушками в скрещенных пучках, гравитационно-оптическими
        ловушками и прочими неудобно работать на нашей экспериментальной установке.

        \subsection{Методы генерации \enquote{бутылочных} ловушек}
        Основная идея генерации - деструктивная интерференция световых пучков. Представим,
        что имеется два когерентных луча разных диаметров сдвинутых по фазе на $\pi$
        друг относительно друга. Если теперь сфокусировать их в одной области, то в силу
        деструктивной интерференции пучок с меньшей перетяжкой \enquote{вырежет} полость в
        пучке с большей перетяжкой, см. рис..

        

        Так как же получить два когерентных пучка, сдвинутых по фазе на $\pi$? Наиболее
        наглядно решение, представленное в работе \cite{isenhower2009atom}. Авторы статьи разбивают
        входной пучок на два, вносят задержку $\pi$ в один из них с помощью
        $\lambda$-пластинки и затем собирают эти пучки в один луч, см. рис.



        Существует альтернативный подход - использование пространственного модулятора света, SLM.
        SLM (от англ. \textit{SLM - Spacial Light Modulator}) - устройство, представляющее собой
        небольшой экран, каждый пиксель которого может вносить заданный фазовый сдвиг в падующую
        на него волну. Таким образом, пространственный модулятор света позволяет контролируемо
        изменять фазовый фронт падающего пучка. Для получения \enquote{бутылочной} ловушки из гауссова
        пучка применяют маску, изображённую на рисунке \ref{fig:SLM_masks}a). Маска - чёрно-белое изображение, выводимое
        на экран SLM, степень черноты определяет величину вносимого фазового сдвига.

        \begin{figure}
            \center
            \includegraphics[width=0.5\textwidth]{slm_masks.jpeg}
            \caption{Маски SLM, использующиеся для получения \enquote{бутылочных} ловушек.}
            \label{fig:SLM_masks}
        \end{figure}

        При фокусировке отражённого от SLM пучка линзой с фокусным расстоянием $f$ образуется
        световое поле с амплитудой $\mathcal{E}(r,z)$. Вводя безразмерные координаты
        $Z = \frac{2\pi}{\lambda}\left(\frac{\sqrt2w}{f}\right)^2(z - f)$ и
        $R = \frac{2\pi}{\lambda}\frac{\sqrt2w}{f}r$, где $w$ - радиус гауссова пучка по уровню
        $\frac1e$, запишем выражение для амплитуды в новых координатах:
        \begin{equation}
            \mathcal{E}(Z, R) \propto \int_0^\infty e^{-\rho'^2}e^{i\frac12\rho'^2Z}
                J_0(R\rho')\rho'd\rho' - 
                2\int_0^B e^{-\rho'^2}e^{i\frac12\rho'^2Z}
                J_0(R\rho')\rho'd\rho',
            \label{equation:field_ampl}
        \end{equation}
        где $J_0$ - функция Бесселя нулевого порядка, $B$ - нормированный на $\sqrt2w$
        радиус круга маски. Формула (\ref{equation:field_ampl}) справедлива вблизи перетяжки,
        $\frac{z - f}{f} \ll 1$, и для узких по сравнению с радиусом линзы пучков. Если пучок
        широкий, бесконечный предел в первом интеграле следует заменить нормированным на $\sqrt2w$
        радиусом линзы.

        \vspace{12pt}

        Авторы работы\cite{jian2002generations} предлагают также несколько усложненный вариант, который
        можно изобразить маской на рисисунке \ref{fig:SLM_masks}b). Если
        сфокусировать пучок отраженный от неё, то вблизи перетяжки поле будет описываться
        аналогичным (\ref{equation:field_ampl}) выражением:
        \begin{equation}
            \mathcal{E}(Z, R) \propto \int_0^\infty e^{-\rho'^2}e^{i\frac12\rho'^2Z}
                J_0(R\rho')\rho'd\rho' - 
                2\int_A^B e^{-\rho'^2}e^{i\frac12\rho'^2Z}
                J_0(R\rho')\rho'd\rho',
            \label{equation:field_ampl_improved}
        \end{equation}
        , где $A$ и $B$ - нормированные на $\sqrt2w$ радиусы внутреннего и внешнего кругов маски соответственно.
        Однако, как будет показано позднее, такой вариант маски при прочих равных создаёт ловушку меньшей глубины.

        На рис. \ref{fig:intensity_b083} приведён результат численного моделирования для параметров маски $A=0,\ B=\sqrt{\ln2}$.

        \begin{figure}
            \center
            \includegraphics[width=0.5\textwidth]{intensity_b083.png}
            \caption{Численный расчёт интенсивности излучения вблизи перетяжки для маски с параметрами $A=0,\ B=\sqrt{\ln2}$.
                Длина волны $\lambda = 852$нм, ширина пучка $w=1.00$мм, фокусное расстояние линзы $\ f=3.00$мм}
            \label{fig:intensity_b083}
        \end{figure}


        \subsection{Техники загрузки атомов}
        В данном разделе будут описаны представленные в литературе способы загрузки атомов в \enquote{синие} ловушки. Начнём с
        работы Roee Ozeri \cite{ozeri1999long}.

        В этой работе авторы размещают \enquote{синюю} ловушку в центре магнито-оптической ловушки, МОТ.
        MOT (от англ. \textit{Magneto-Optical Trap}) замедляет быстрые атомы с помощью специального
        охлаждающего лазера и удерживает их за счёт градиента магнитного поля. Находясь в градиенте
        магнитного поля, атом испытывает силу $F \propto -m\grad{H}$, где $m$ - магнитное квантовое число,
        $H$ - проекция напряжённости магнитного поля на ось квантования. Параметры охлаждающего луча подобраны
        таким образом, что при движении по $\grad{H}$ атом переводится в состояние с $m > 0$ и его начинает
        тянуть в противоположную сторону. При движением же против $\grad{H}$, атом переводится в
        состояние с $m < 0$ и его снова начинает тянет назад. 
        
        Сама загрузка делится на четыре этапа. На первом $\approx4\cross10^8$ атомов рубидия набираются в МОТ в
        течение 500 мсек. На втором приглушаются охлаждающие пучки МОТ, увеличивается их отстройка. Также
        увеличивается отстройка лазера перекачки (это ещё один лазер, необходимый для работы MOT). Спустя
        30 мсек, распределение плотности облака атомов приобретает гауссову форму. Третий этап - отключение
        магнитного поля. И, наконец, на последнем этапе отключается лазер перекачки. В течение всех этапов
        лазер \enquote{синей} ловушки остаётся включенным.

        В другой работе\cite{isenhower2009atom} представлен вариант попроще: МОТ набирается в течение 1 секунды,
        после чего включается лазер \enquote{синей} ловушки, фокусирущийся в центре МОТ, и спустя 1 мсек,
        выключаются лазеры МОТ. Через время $t$ МОТ опять включается для того , чтобы увидеть флуоресценцию
        пленённых атомов.

        В третьей статье\cite{xu2010trapping} ловушка как и прежде размещается в центре МОТ. Чтобы атом смог попасть
        в ловушку, он должен преодолеть потенциал, окружающий её центр. Однако для этого атом должен обладать
        достаточной кинетической энергией, которую к тому же ему придётся \enquote{растерять}, чтобы, попав в центр ловушки,
        он не вылетел из неё. Для этого \enquote{синий} лазер на 1 мсек отключают (избавляются от проблемного потенциала).
        Затем его включают и 39 мсек набирают сигнал флуоресценции. По окончании такого сорокамиллисекндного цикла
        по полученному сигналу флуоресценции можно определить, есть ли атом в ловушке. Если его там не оказалось,
        то цикл повторяют.

    
    \section{Оригинальная часть}
        Здесь будет описана постановка задачи и этапы её решения.

        \subsection{Постановка задачи}
        Необходимо добится генерации \enquote{бутылочной} ловушки и загрузки в неё атома. Также в случае успеха
        требуется определить время декогеренции - ожидается, что это время будет больше, чем таковое для
        \enquote{красной} ловушки.

        Мотивацией для работы с \enquote{синими} ловушками служит в частности тот факт, что использующийся нами для
        \enquote{красных} ловушек лазер содержит в своём спектре паразитные частоты ОВЧ-диапазона, которые предположительно
        вызывают ускоренный распад состояния $F=2$.

        \subsection{Результаты моделирования}
        \subsubsection{Форма \enquote{бутылочной} ловушки}
        Первым делом предстояло провести моделирование. В качестве рассчётной формулы была взята формула
        (\ref{equation:field_ampl_improved}):
        \begin{equation*}
            \mathcal{E}(Z, R) \propto \int_0^\infty e^{-\rho'^2}e^{i\frac12\rho'^2Z}
            J_0(R\rho')\rho'd\rho' - 
            2\int_A^B e^{-\rho'^2}e^{i\frac12\rho'^2Z}
            J_0(R\rho')\rho'd\rho',
        \end{equation*}

        Ниже приведён график численных расчётов для параметров $A = 0.3, B = 0.9$.
        \begin{figure}
            \center
            \includegraphics[width=0.9\textwidth]{bob_b09a03.png}
            \caption{Распределение интенсивности излучения в \enquote{бутылочной} ловушке для параметров
                $A = 0.3, B = 0.9$}
            \label{fig:BOB a=0.3, b=0.9}
        \end{figure}
        Нетрудно видеть, что глубина ловушки зависит от направления движения фтома в ней: если атом движется вдоль
        оси $Oz$, то глубина максимальна, если атом движется перпендикулярно $Oz$, то глубина принимает среднее значение,
        и, наконец, если атом движется под особым углом, - в направлении на седловую точку, - то глубина минимальна.
        Это особое направление будем называть конусом убегания.

        Какое же соотношение параметров $A$ и $B$ обеспечит наибольшую глубину? Прежде чем дать ответ, замечу, что
        помимо максимальной глубины нужно добиваться минимального значения интенсивности в центре ловушки. Нетрудно показать,
        что ноль интенсивности в центре достигается при $e^{-A^2} - e^{-B^2}= \frac12$. Вернёмся к вопросу о максимизации
        глубины: при помощи формулы из предыдущего предложения, численно была найдена зависимость глубины ловушки
        в направлении конуса убегания от параметра $B$.

        \begin{figure}
            \center
            \includegraphics[width=0.8\textwidth]{depth.png}
            \caption{Зависимость минимальной глубины ловушки от параметра $B$. Параметр $A$ находился из связи
            $e^{-A^2} - e^{-B^2}= \frac12$, полученной из условия обнуления интенсивности в центре ловушки.}
            \label{fig:depth}
        \end{figure}

        Из приведённой на рис.\ref{fig:depth} зависимости следует, что параметр $B$ следует выбирать либо
        как можно меньшим, - а из связи $A$ и $B$ и их неотрицательности следует, что
        $\min{(B)}=\sqrt{\ln2} \approx 0.83$, - либо это $B$ следует брать как можно большим. Оба случая описывают одну
        ситуацию, когда маска фактически состоит из двух областей (см. рис.\ref{fig:SLM_masks}(а)). Действительно, при $B=\sqrt{\ln2}$ $A=0$,
        а при $B \rightarrow \infty$ $A \rightarrow \sqrt{\ln2}$. Ниже приведено распределение интенсивности для случая $B=\sqrt{\ln2}$.

        \begin{figure}
            \center
            \includegraphics[width=0.8\textwidth]{bob_b083.png}
            \caption{Распределение интенсивности излучения в \enquote{бутылочной} ловушке для параметров
            $A = 0, B = \sqrt{\ln2}$.}
            \label{fig:BOB a=0 b=sqrt(ln2)}
        \end{figure}

        %===========================================================================================
        \subsubsection{Изображение ловушки}
        
        Для экспериментального измерения формы \enquote{бутылочной} ловушки была использована схема,
        изображенная на рис.\ref{fig:BOB form measurement}.
        \begin{figure}
            \center
            \includegraphics[width=1\textwidth]{imaging_scheme.pdf}
            \caption{Схема установки, использовавшейся для определения распределения интенсивности в
            \enquote{бутылочной} ловушке (\textit{BOB} - Bottle Beam).}
            \label{fig:BOB form measurement}
        \end{figure}
        Полученные экспериментальные данные приведены на рис.\ref{fig:exp BOB form}.
        \begin{figure}
            \center
            \includegraphics{exp_bob_combined.jpeg}
            \caption{Распределения интенсивности в ловушке: слева - продольное, справа - поперечное.
            По оси $Oz$ отложено смещение камеры, по осям $Ox, Oy$ и $Or$ - размеры пучка на
            матрице камеры.}
            \label{fig:exp BOB form}
        \end{figure}
        Помимо искривления, которое можно списать на несовпадение оси пучка и оптической оси системы,
        ловушка обладает асимметрией по оси $Oz$. Эту асимметрию можно объяснить сферическими
        аберрациями в изображающей системе.
        
Чтобы убедиться в этом, была разработана программа на языке OpenCL, реализующая численное
        моделирвание распространения пучка. Результаты моделирования представлены на рисунке
        \ref{fig:abber compare}.

        \begin{figure}
            \center
            \includegraphics{20.48UMx20.48UM.jpeg}
            \caption{Сравнение двух \enquote{бутылочных}. Слева изображена ловушка, полученная в системе
            без аббераций, справа - с учётом сферических аббераций.}
            \label{fig:abber compare}
        \end{figure}
        %===========================================================================================

        \subsection{Экспериментальая установка}


        \subsection{Эксперимент}
    
    \section{Выводы}
        Выводы

    \bibliography{ref.bib}

\end{document}
